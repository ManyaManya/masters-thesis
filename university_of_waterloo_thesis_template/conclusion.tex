%======================================================================
\chapter{Conclusion}
%======================================================================
This work studies variation in the refractive index contrast of HCPCFs and their use as a liquid-waveguide. We were able to demonstrate the preservation of light bandgap guidance through liquid by experimentally confirm the scaling laws for liquid-filled fibers of H${}_2$O and D${}_2$O.  The main results of the thesis culminate in measuring the interaction of suspended ICG particles within the mode of the fiber, demonstrating the potential of creating fluorescent light sources out of suspended particles in liquid core fibers.\\
 
The unique optical properties of semiconducting CNTs are introduced and an argument for their suitability as a fluorescent light source is explored. For common commercially available HCPCF, 1550nm HCPCF is the most practical as it has the only bandgap that overlaps with the absorption and emission spectrum of several chiralities.\\

While the results of suspended ICG particles are promising and show a proof-of-concept, the progression to suspended CNTs in HCPCF is incomplete.The initial tests with a sorted CNTs sample were inconclusive and more time needs to be spent on producing a high-fluorescence  sample. More work needs to be done experimentally with the CNTs and continuing efforts will hopefully lead to uses in fiber-integrated devices. Besides the CNT samples, one of the main challenges that remains is consistent and high coupling to the 1550nm HCPCF. Butt-coupled mechanical splicing chips fabricated for 800nm HCPCF consistently achieve coupling around 75-80\% and robustly hold the fibers together\cite{maruf}, but the same recipe does not produce high-fidelity butt-coupled mechanical splicing chips for 1550nm HCPCF. This is another avenue that needs to be explored but is a very solvable problem.
