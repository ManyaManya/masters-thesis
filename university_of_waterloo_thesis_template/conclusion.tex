%======================================================================
\chapter{Conclusion}
%======================================================================
In this thesis we studied variations in the refractive index contrast of HCPCFs and their use as a liquid-waveguide. We were able to demonstrate the preservation of light bandgap guidance through liquid by experimentally confirming the scaling laws for liquid-filled fibers of H${}_2$O and D${}_2$O.
The most important results of the thesis were obtained in the measurements of the interaction of suspended ICG particles within the mode of the fiber, where we demonstrated the potential of creating fluorescent light sources out of suspended particles in liquid core fibers.\\

In our analysis, we described the unique optical properties of semiconducting CNTs, and we explored an argument for their suitability as a fluorescent light source. We found that that for common commercially available HCPCF, 1550nm HCPCF is the most practical fiber, because it has the only bandgap that overlaps with the absorption and emission spectrum of several chiralities.\\

While the results of suspended ICG particles are promising and show a proof-of-concept, the progression to suspended CNTs in HCPCF is incomplete. The initial tests with a sorted CNTs sample were inconclusive and further inverstigations are needed to produce a high-fluorescence sample. Future experimental efforts will hopefully lead to uses of CNTs in fiber-integrated devices. Besides the CNT samples, one of the main challenges that remains is obtaining a consistent and high coupling to the 1550nm HCPCF. Butt-coupled mechanical splicing chips fabricated for 800nm HCPCF consistently achieve coupling around 75-80\% and robustly hold the fibers together\cite{maruf}, but the same recipe does not produce high-fidelity butt-coupled mechanical splicing chips for 1550nm HCPCF. This is another avenue that needs to be explored but we are confident that it can be solved with modest efforts.\\
