%======================================================================
\chapter{Conclusion}
%======================================================================
In this thesis use of  HCPCFs as a liquid-waveguide was studied. The preservation of light bandgap guidance through liquid demonstrated by experimentally confirming the scaling laws for liquid-filled fibers of H${}_2$O and D${}_2$O.
The main results of the thesis were obtained in the measurements of the interaction of suspended ICG particles within the mode of the fiber. The dye molecules were excited in the fiber and fully-filled HCPCF guiding light via optical bandgap exhibited a higher efficiency than that seen in the liquid-core fiber. Despite difficulty with the coupling, there is high enough efficiency the potential of creating fluorescent light sources. However, using organic molecules becomes troublesome due to rapid degradation and photobleaching. Semiconducting CNTs in theory can also be used as a fluorescent medium, removing the lifespan issues of organic dyes and and increased tunability through chirality selection processes. The overlap in certain chirality CNT spectrums and the bandgap of commercially available HCPCF when filled with heavy water makes further study into their integration warranted. 