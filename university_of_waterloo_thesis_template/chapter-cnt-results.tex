\subsection{Experiment Set-Up}
\begin{figure}[htb!]
	\foreach \x \y in {CNT\_Abs\_SetUp/0.6, pickoff/0.4,CNT\_Fluor\_SetUp/0.6}
	{ 
		\begin{subfigure}[b]{\y\textwidth}
			\includegraphics[width=\textwidth]{./Figures/CNT_Measured/\x.png}
			\caption{}
		\end{subfigure}
		\hfil
	}
	\caption{(a) (b)fwhm=6nm (c)}
	\label{fig:cnt_setup}
\end{figure}

\subsubsection{Sample History}
CNT samples were prepared by HeeBong Yang from the QuIN Lab at the University of Waterloo. SG65i powder was purchased from Sigma Aldrich and Dispersed in a surfactant at an initial powder concentration of 1 mg/mL. The sample then underwent a procedure of purification steps, sorting with polymers \& surfactants, and polymer exchange. The final condition of the sample was 65\% (7,5), (7,6) dominant SWCNTs in DI water with 0.04\% DOC, but at an unknown concentration.
\subsubsection{Sample Characteristics}
CNT solutions follow Beer-Lambert's Law \cite{schoppler, jeong}, $A = log(\frac{I_{in}}{I_{out}}) = \varepsilon CL$ so the concentration can be deduced from the measured absorbance. Fig.\ref{fig:cnt_abs}(a) and the average previously reported extinction coefficient\cite{blanch, anson, jeong} $\varepsilon= 30.98$ mL mg${}^{-1}$cm${}^{-1}$  estimate a sample concentration around $0.0042\pm 0.0007$ mg/mL.

\begin{figure}[h]
	\centering
	\foreach \x in {OD, time}
	{ 
		\begin{subfigure}[b]{0.45\textwidth}
			\includegraphics[width=\textwidth]{./Figures/CNT_Measured/\x.png}
			\caption{}
		\end{subfigure}
		\hfil
	}
	\caption{(a) Absorbance spectrum of CNT sorted CNT sample (b)Absorbance of CNT sorted sample over 60 minutes, very low photobleaching.}
	\label{fig:cnt_abs}
\end{figure}

Unfortunately no PL was detected from the sample. There is 23.25\% pm 10\% decrease in PL intensity when using H${}_2$O instead of D${}_2$O for (7, 5) and 42.5\% pm 5\% for (7,6) \%\cite{wei}. In combination with a 48$\mu$W source when currently QY are expected to be around (7, 5) 1.04 ± 0.10\% , (7, 6) 1.40 ± 0.09 could explain the lack of detection. Trying thinner cuvette, diluting the sample as self-quenching is also common, and preparation of a D${}_2$O sample. 
\subsection{Expected Metrics}
SWNT Fluorescence Action Cross Sections  (7, 5) 3.2 $(10^{-19} (cm^2/mol C)$(7, 6)2.3$(10^{-19} (cm^2/mol C)$ \cite{tsyboulski}
-optical density
-expected efficiency
\clearpage