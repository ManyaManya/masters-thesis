\begin{thebibliography}{99}
%HCPCF Background
\bibitem{yariv} Yariv, Amnon and Pochi Albert Yeh, Optical Waves in Crystals: Propagation and Control of Laser Radiation,”1st ed. Wiley-Interscience, 2002.
\bibitem{joannopoulos} J. D. Joannopoulos, Photonic crystals: molding the flow of light, 2nd ed. Princeton: Princeton University Press, 2008.
\bibitem{villeneuve} P. R. Villeneuve and M. Piche´, “Photonic band gaps in two-dimensional square and hexagonal lattices,” Phys. Rev. B, vol. 46, no. 8, pp. 4969–4972, Aug. 1992, doi: 10.1103/PhysRevB.46.4969.
\bibitem{sukhoivanov} I. A. Sukhoivanov and I. V. Guryev, Photonic Crystals: Physics and Practical Modeling, vol. 152. Berlin, Heidelberg: Springer Berlin Heidelberg, 2009. doi: 10.1007/978-3-642-02646-1.
\bibitem{birks} T. A. Birks, D. M. Bird, T. D. Hedley, J. M. Pottage, and P. St. J. Russell, “Scaling laws and vector effects in bandgap-guiding fibres,” Opt. Express, vol. 12, no. 1, p. 69, 2004, doi: 10.1364/OPEX.12.000069.
\bibitem{antonopoulos} G. Antonopoulos, F. Benabid, T. A. Birks, D. M. Bird, J. C. Knight, and P. St. J. Russell, “Experimental demonstration of the frequency shift of bandgaps in photonic crystal fibers due to refractive index scaling,” Opt. Express, vol. 14, no. 7, p. 3000, 2006, doi: 10.1364/OE.14.003000.
%fiberfilling
\bibitem{maruf} R. A. Maruf and M. Bajcsy, “On-chip splicer for coupling light between
photonic crystal and solid-core fibers,” Appl. Opt., vol. 56, no. 16, p.
4680, Jun. 2017.
\bibitem{xiao} L. Xiao, W. Jin, M. S. Demokan, H. L. Ho, Y. L. Hoo, and C. Zhao,
“Fabrication of selective injection microstructured optical fibers with a
conventional fusion splicer,” Opt. Express, vol. 13, no. 22, p. 9014, 2005.
\bibitem{kedenburg} S. Kedenburg, M. Vieweg, T. Gissibl, and H. Giessen, “Linear refractive
index and absorption measurements of nonlinear optical liquids in the
visible and near-infrared spectral region,” Opt. Mater. Express, vol. 2,
no. 11, p. 1588, Nov. 201

%CNT Background
\bibitem{dresselhaus} S. Dresselhaus, “PHYSICS OF CARBON NANOTUBES,” Carbon, 33(7), 883-891, 1995.
\bibitem{popov} V. Popov, “Carbon nanotubes: properties and application,” Materials Science and Engineering: R: Reports, vol. 43, no. 3, pp. 61–102, Jan. 2004.
\bibitem{yamashita} S. Yamashita, “Nonlinear optics in carbon nanotube, graphene, and related 2D materials,” APL Photonics, vol. 4, no. 3, p. 034301, Mar. 2019.
\bibitem{saito}R. Saito, M. Fujita, G. Dresselhaus, and M. S. Dresselhaus, “Electronic structure of chiral graphene tubules,” Appl. Phys. Lett., vol. 60, no. 18, pp. 2204–2206, May 1992.
\bibitem{thomsen}C. Thomsen, S. Reich, and J. Maultzsch, Carbon Nanotubes: Basic Concepts and Physical Properties, 1st ed. Wiley, 2004. 
\bibitem{kataura} H. Kataura et al., “Optical properties of single-wall carbon nanotubes,” Synthetic Metals, vol. 103, no. 1–3, pp. 2555–2558, Jun. 1999.
\bibitem{yamashita.tutorial} S. Yamashita, “A Tutorial on Nonlinear Photonic Applications of Carbon Nanotube and Graphene,” J. Lightwave Technol., vol. 30, no. 4, pp. 427–447, Feb. 2012.
\bibitem{gambetta}A. Gambetta et al., “Sub-100 fs two-color pump-probe spectroscopy of Single Wall Carbon Nanotubes with a 100 MHz Er-fiber laser system,” p. 8, 2008.
\bibitem{weisman} R. B. Weisman and S. M. Bachilo, “Dependence of Optical Transition Energies on Structure for Single-Walled Carbon Nanotubes in Aqueous Suspension: An Empirical Kataura Plot,” Nano Lett., vol. 3, no. 9, pp. 1235–1238, Sep. 2003.
\bibitem{bachilo} S. M. Bachilo, M. S. Strano, C. Kittrell, R. H. Hauge, R. E. Smalley, and R. B. Weisman, “Structure-Assigned Optical Spectra of Single-Walled Carbon Nanotubes,” Science, vol. 298, no. 5602, pp. 2361–2366, Dec. 2002. 
\bibitem{giordani}S. Giordani et al., “Debundling of Single-Walled Nanotubes by Dilution: Observation of Large Populations of Individual Nanotubes in Amide Solvent Dispersions,” J. Phys. Chem. B, vol. 110, no. 32, pp. 15708–15718, Aug. 2006.
\bibitem{martinez} A. Martinez and S. Yamashita, “Carbon Nanotube-Based Photonic Devices: Applications in Nonlinear Optics,” In: J. M. Marulanda, Ed., Carbon Nanotubes Applications on Electron Devices, InTech, 2011. 
\bibitem{margulis} Vl. A. Margulis and T. A. Sizikova, “Theoretical study of third-order nonlinear optical response of semiconductor carbon nanotubes,” Physica B: Condensed Matter, vol. 245, no. 2, pp. 173–189, Mar.1998.
\bibitem{turek} E. Turek, T. Shiraki, T. Shiraishi, T. Shiga, T. Fujigaya, and D. Janas, “Single-step isolation of carbon nanotubes with narrow-band light emission characteristics,” Sci Rep, vol. 9, no. 1, p. 535, Dec. 2019

%Modecoupling
\bibitem{bajcsy} M. Bajcsy et al., “Laser-cooled atoms inside a hollow-core photonic-crystal fiber,” Phys. Rev. A, vol. 83, no. 6, p. 063830, Jun. 2011, doi: 10.1103/PhysRevA.83.063830. 
\bibitem{hilton} A. P. Hilton, C. Perrella, F. Benabid, B. M. Sparkes, A. N. Luiten, and P. S. Light, “High-efficiency cold-atom transport into a waveguide trap,” Phys. Rev. Applied, vol. 10, no. 4, p. 044034, Oct. 2018, doi: 10.1103/PhysRevApplied.10.044034. A
\bibitem{domokos} P. Domokos, P. Horak, and H. Ritsch, “Quantum description of light-pulse scattering on a single atom in waveguides,” Phys. Rev. A, vol. 65, no. 3, p. 033832, Mar. 2002, doi: 10.1103/PhysRevA.65.033832.
\bibitem{mazoni} M. T. Manzoni, “New Systems for Quantum Nonlinear Optics,”. 2017, Thesis, p. 39-40.

%ICG
	\bibitem{Fang} X. Fang et al., “One-step condensation synthesis and characterizations of indocyanine green,” Results in Chemistry, vol. 3, p. 100092, Jan. 2021, doi: 10.1016/j.rechem.2020.100092.
\bibitem{holzer}W. Holzer et al., "Photostability and thermal stability of indocyanine green," Journal of Photochemistry and Photobiology B: Biology,  vol. 47, no. 2-3, pp. 155-164, Dec. 1998.
\bibitem{landsman}M. L. Landsman, G. Kwant, G. A. Mook, and W. G. Zijlstra, “Light-absorbing properties, stability, and spectral stabilization of indocyanine green,” Journal of Applied Physiology, vol. 40, no. 4, pp. 575–583, Apr. 1976.
\bibitem{mauerer}M. Mauerer, A. Penzkofer, and J. Zweck, “Dimerization, J-aggregation and J-disaggregation dynamics of indocyanine green in heavy water,” Journal of Photochemistry and Photobiology B: Biology, vol. 47, no. 1, pp. 68–73, Nov. 1998.
\bibitem{rotermund}F. Rotermund, R. Weigand, W. Holzer, M. Wittmann, and A. Penzkofer, “Fluorescence spectroscopic analysis of indocyanine green J aggregates in water,” Journal of Photochemistry and Photobiology A: Chemistry, vol. 110, no. 1, pp. 75–78, Oct. 1997.
\bibitem{philip}R. Philip, A. Penzkofer, W. Bäumler, R. M. Szeimies, and C. Abels, “Absorption and fluorescence spectroscopic investigation of indocyanine green,” Journal of Photochemistry and Photobiology A: Chemistry, vol. 96, no. 1–3, pp. 137–148, May 1996.
\bibitem{saxena}V. Saxena, M. Sadoqi, and J. Shao, “Degradation Kinetics of Indocyanine Green in Aqueous Solution,” Journal of Pharmaceutical Sciences, vol. 92, no. 10, pp. 2090–2097, Oct. 2003. 
\bibitem{farrakhova}D. Farrakhova et al., “Fluorescence imaging analysis of distribution of indocyanine green in molecular and nanoform in tumor model,” Photodiagnosis and Photodynamic Therapy, vol. 37, p. 102636, Mar. 2022, doi: 10.1016/j.pdpdt.2021.102636.
\bibitem{spartalis} E. Spartalis et al., “Intraoperative Indocyanine Green (ICG) Angiography for the Identification of the Parathyroid Glands: Current Evidence and Future Perspectives,” In Vivo, vol. 34, no. 1, pp. 23–32, 2020, doi: 10.21873/invivo.11741.
\bibitem{dedora}D. J. DeDora et al., “Sulfobutyl ether $\beta$-cyclodextrin and methyl $\beta$-cyclodextrin enhance and stabilize fluorescence of aqueous indocyanine green: Sulfobutyl Ether $\beta$-Cyclodextrin and METHYL $\beta$-Cyclodextrin,” J. Biomed. Mater. Res., vol. 104, no. 7, pp. 1457–1464, Oct. 2016, doi: 10.1002/jbm.b.33496.
\bibitem{weigand}R. Weigand, F. Rotermund, and A. Penzkofer, “Degree of aggregation of indocyanine green in aqueous solutions determined by Mie scattering,” Chemical Physics, vol. 220, no. 3, pp. 373–384, Aug. 1997, doi: 10.1016/S0301-0104(97)00150-X.
\end{thebibliography}
