\begin{thebibliography}{99}
%HCPCF Background
\bibitem{yariv} Yariv, Amnon and Pochi Albert Yeh, Optical Waves in Crystals: Propagation and Control of Laser Radiation,”1st ed. Wiley-Interscience, 2002.
\bibitem{joannopoulos} J. D. Joannopoulos, Photonic crystals: molding the flow of light, 2nd ed. Princeton: Princeton University Press, 2008.
\bibitem{chourasia}R. K. Chourasia and V. Singh, “Estimation of photonic band gap in the hollow core cylindrical multilayer structure,” Superlattices and Microstructures, vol. 116, pp. 191–199, Apr. 2018, doi: 10.1016/j.spmi.2018.02.023.
\bibitem{villeneuve} P. R. Villeneuve and M. Piche´, “Photonic band gaps in two-dimensional square and hexagonal lattices,” Phys. Rev. B, vol. 46, no. 8, pp. 4969–4972, Aug. 1992, doi: 10.1103/PhysRevB.46.4969.
\bibitem{cregan}R. F. Cregan et al., “Single-Mode Photonic Band Gap Guidance of Light in Air,” Science, vol. 285, no. 5433, pp. 1537–1539, Sep. 1999, doi: 10.1126/science.285.5433.1537.
\bibitem{sukhoivanov} I. A. Sukhoivanov and I. V. Guryev, Photonic Crystals: Physics and Practical Modeling, vol. 152. Berlin, Heidelberg: Springer Berlin Heidelberg, 2009. doi: 10.1007/978-3-642-02646-1.
\bibitem{birks} T. A. Birks, D. M. Bird, T. D. Hedley, J. M. Pottage, and P. St. J. Russell, “Scaling laws and vector effects in bandgap-guiding fibres,” Opt. Express, vol. 12, no. 1, p. 69, 2004, doi: 10.1364/OPEX.12.000069.
\bibitem{antonopoulos} G. Antonopoulos, F. Benabid, T. A. Birks, D. M. Bird, J. C. Knight, and P. St. J. Russell, “Experimental demonstration of the frequency shift of bandgaps in photonic crystal fibers due to refractive index scaling,” Opt. Express, vol. 14, no. 7, p. 3000, 2006, doi: 10.1364/OE.14.003000.
%fiberfilling
\bibitem{maruf} R. A. Maruf and M. Bajcsy, “On-chip splicer for coupling light between
photonic crystal and solid-core fibers,” Appl. Opt., vol. 56, no. 16, p.
4680, Jun. 2017.
\bibitem{xiao} L. Xiao, W. Jin, M. S. Demokan, H. L. Ho, Y. L. Hoo, and C. Zhao,
“Fabrication of selective injection microstructured optical fibers with a
conventional fusion splicer,” Opt. Express, vol. 13, no. 22, p. 9014, 2005.
\bibitem{kedenburg} S. Kedenburg, M. Vieweg, T. Gissibl, and H. Giessen, “Linear refractive
index and absorption measurements of nonlinear optical liquids in the
visible and near-infrared spectral region,” Opt. Mater. Express, vol. 2,
no. 11, p. 1588, Nov. 201

%Modecoupling
\bibitem{bajcsy} M. Bajcsy et al., “Laser-cooled atoms inside a hollow-core photonic-crystal fiber,” Phys. Rev. A, vol. 83, no. 6, p. 063830, Jun. 2011, doi: 10.1103/PhysRevA.83.063830. 
\bibitem{hilton} A. P. Hilton, C. Perrella, F. Benabid, B. M. Sparkes, A. N. Luiten, and P. S. Light, “High-efficiency cold-atom transport into a waveguide trap,” Phys. Rev. Applied, vol. 10, no. 4, p. 044034, Oct. 2018, doi: 10.1103/PhysRevApplied.10.044034. A
\bibitem{domokos} P. Domokos, P. Horak, and H. Ritsch, “Quantum description of light-pulse scattering on a single atom in waveguides,” Phys. Rev. A, vol. 65, no. 3, p. 033832, Mar. 2002, doi: 10.1103/PhysRevA.65.033832.
\bibitem{mazoni} M. T. Manzoni, “New Systems for Quantum Nonlinear Optics,”. 2017, Thesis, p. 39-40.

%ICG
\bibitem{Fang} X. Fang et al., “One-step condensation synthesis and characterizations of indocyanine green,” Results in Chemistry, vol. 3, p. 100092, Jan. 2021, doi: 10.1016/j.rechem.2020.100092.
\bibitem{holzer}W. Holzer et al., "Photostability and thermal stability of indocyanine green," Journal of Photochemistry and Photobiology B: Biology,  vol. 47, no. 2-3, pp. 155-164, Dec. 1998.
\bibitem{landsman}M. L. Landsman, G. Kwant, G. A. Mook, and W. G. Zijlstra, “Light-absorbing properties, stability, and spectral stabilization of indocyanine green,” Journal of Applied Physiology, vol. 40, no. 4, pp. 575–583, Apr. 1976.
\bibitem{mauerer}M. Mauerer, A. Penzkofer, and J. Zweck, “Dimerization, J-aggregation and J-disaggregation dynamics of indocyanine green in heavy water,” Journal of Photochemistry and Photobiology B: Biology, vol. 47, no. 1, pp. 68–73, Nov. 1998.
\bibitem{rotermund}F. Rotermund, R. Weigand, W. Holzer, M. Wittmann, and A. Penzkofer, “Fluorescence spectroscopic analysis of indocyanine green J aggregates in water,” Journal of Photochemistry and Photobiology A: Chemistry, vol. 110, no. 1, pp. 75–78, Oct. 1997.
\bibitem{philip}R. Philip, A. Penzkofer, W. Bäumler, R. M. Szeimies, and C. Abels, “Absorption and fluorescence spectroscopic investigation of indocyanine green,” Journal of Photochemistry and Photobiology A: Chemistry, vol. 96, no. 1–3, pp. 137–148, May 1996.
\bibitem{saxena}V. Saxena, M. Sadoqi, and J. Shao, “Degradation Kinetics of Indocyanine Green in Aqueous Solution,” Journal of Pharmaceutical Sciences, vol. 92, no. 10, pp. 2090–2097, Oct. 2003. 
\bibitem{farrakhova}D. Farrakhova et al., “Fluorescence imaging analysis of distribution of indocyanine green in molecular and nanoform in tumor model,” Photodiagnosis and Photodynamic Therapy, vol. 37, p. 102636, Mar. 2022, doi: 10.1016/j.pdpdt.2021.102636.
\bibitem{spartalis} E. Spartalis et al., “Intraoperative Indocyanine Green (ICG) Angiography for the Identification of the Parathyroid Glands: Current Evidence and Future Perspectives,” In Vivo, vol. 34, no. 1, pp. 23–32, 2020, doi: 10.21873/invivo.11741.
\bibitem{dedora}D. J. DeDora et al., “Sulfobutyl ether $\beta$-cyclodextrin and methyl $\beta$-cyclodextrin enhance and stabilize fluorescence of aqueous indocyanine green: Sulfobutyl Ether $\beta$-Cyclodextrin and METHYL $\beta$-Cyclodextrin,” J. Biomed. Mater. Res., vol. 104, no. 7, pp. 1457–1464, Oct. 2016, doi: 10.1002/jbm.b.33496.
\bibitem{weigand}R. Weigand, F. Rotermund, and A. Penzkofer, “Degree of aggregation of indocyanine green in aqueous solutions determined by Mie scattering,” Chemical Physics, vol. 220, no. 3, pp. 373–384, Aug. 1997, doi: 10.1016/S0301-0104(97)00150-X.
\bibitem{hiemenz} P. C. Hiemenz and R. D. Vold, “Particle size from the optical properties of flocculating carbon dispersions,” Journal of Colloid and Interface Science, vol. 21, no. 5, pp. 479–488, May 1966, doi: 10.1016/0095-8522(66)90046-8.

%Liquid waveguides
\bibitem{cox}A. J. Cox, A. J. DeWeerd, and J. Linden, “An experiment to measure Mie and Rayleigh total scattering cross sections,” American Journal of Physics, vol. 70, no. 6, pp. 620–625, Jun. 2002, doi: 10.1119/1.1466815.
\bibitem{vezenov}D. V. Vezenov, B. T. Mayers, D. B. Wolfe, and G. M. Whitesides, “Integrated fluorescent light source for optofluidic applications,” Appl. Phys. Lett., vol. 86, no. 4, p. 041104, Jan. 2005, doi: 10.1063/1.1850610.
\bibitem{bliss}C. L. Bliss, J. N. McMullin, and C. J. Backhouse, “Integrated wavelength-selective optical waveguides for microfluidic-based laser-induced fluorescence detection,” Lab Chip, vol. 8, no. 1, pp. 143–151, 2008, doi: 10.1039/B711601B.
\bibitem{conroy}R. S. Conroy, B. T. Mayers, D. V. Vezenov, D. B. Wolfe, M. G. Prentiss, and G. M. Whitesides, “Optical waveguiding in suspensions of dielectric particles,” p. 5.
%CNT Background
\bibitem{cusano}A. Cusano et al., “Optical probes based on optical fibers and single-walled carbon nanotubes for hydrogen detection at cryogenic temperatures,” Appl. Phys. Lett., vol. 89, no. 20, p. 201106, Nov. 2006, doi: 10.1063/1.2370292.
\bibitem{dresselhaus} S. Dresselhaus, “PHYSICS OF CARBON NANOTUBES,” Carbon, 33(7), 883-891, 1995.
\bibitem{popov} V. Popov, “Carbon nanotubes: properties and application,” Materials Science and Engineering: R: Reports, vol. 43, no. 3, pp. 61–102, Jan. 2004.
\bibitem{yamashita} S. Yamashita, “Nonlinear optics in carbon nanotube, graphene, and related 2D materials,” APL Photonics, vol. 4, no. 3, p. 034301, Mar. 2019.
\bibitem{saito}R. Saito, M. Fujita, G. Dresselhaus, and M. S. Dresselhaus, “Electronic structure of chiral graphene tubules,” Appl. Phys. Lett., vol. 60, no. 18, pp. 2204–2206, May 1992.
\bibitem{thomsen}C. Thomsen, S. Reich, and J. Maultzsch, Carbon Nanotubes: Basic Concepts and Physical Properties, 1st ed. Wiley, 2004. 
\bibitem{kataura} H. Kataura et al., “Optical properties of single-wall carbon nanotubes,” Synthetic Metals, vol. 103, no. 1–3, pp. 2555–2558, Jun. 1999.
\bibitem{saito} R. Saito, G. Dresselhaus, and M. S. Dresselhaus, “Trigonal warping effect of carbon nanotubes,” Phys. Rev. B, vol. 61, no. 4, pp. 2981–2990, Jan. 2000, doi: 10.1103/PhysRevB.61.2981.
\bibitem{maruyama} S. Maruyama, Fullerene and Carbon Nanotube Site [Online]. Available: http://www.photon.t.u-tokyo.ac.jp/~maruyama/nanotube.html
\bibitem{yamashita.tutorial} S. Yamashita, “A Tutorial on Nonlinear Photonic Applications of Carbon Nanotube and Graphene,” J. Lightwave Technol., vol. 30, no. 4, pp. 427–447, Feb. 2012.
\bibitem{gambetta}A. Gambetta et al., “Sub-100 fs two-color pump-probe spectroscopy of Single Wall Carbon Nanotubes with a 100 MHz Er-fiber laser system,” p. 8, 2008.
\bibitem{weisman} R. B. Weisman and S. M. Bachilo, “Dependence of Optical Transition Energies on Structure for Single-Walled Carbon Nanotubes in Aqueous Suspension: An Empirical Kataura Plot,” Nano Lett., vol. 3, no. 9, pp. 1235–1238, Sep. 2003.
\bibitem{bachilo} S. M. Bachilo, M. S. Strano, C. Kittrell, R. H. Hauge, R. E. Smalley, and R. B. Weisman, “Structure-Assigned Optical Spectra of Single-Walled Carbon Nanotubes,” Science, vol. 298, no. 5602, pp. 2361–2366, Dec. 2002. 
\bibitem{giordani}S. Giordani et al., “Debundling of Single-Walled Nanotubes by Dilution: Observation of Large Populations of Individual Nanotubes in Amide Solvent Dispersions,” J. Phys. Chem. B, vol. 110, no. 32, pp. 15708–15718, Aug. 2006.
\bibitem{martinez} A. Martinez and S. Yamashita, “Carbon Nanotube-Based Photonic Devices: Applications in Nonlinear Optics,” In: J. M. Marulanda, Ed., Carbon Nanotubes Applications on Electron Devices, InTech, 2011. 
\bibitem{margulis} Vl. A. Margulis and T. A. Sizikova, “Theoretical study of third-order nonlinear optical response of semiconductor carbon nanotubes,” Physica B: Condensed Matter, vol. 245, no. 2, pp. 173–189, Mar.1998.
\bibitem{turek} E. Turek, T. Shiraki, T. Shiraishi, T. Shiga, T. Fujigaya, and D. Janas, “Single-step isolation of carbon nanotubes with narrow-band light emission characteristics,” Sci Rep, vol. 9, no. 1, p. 535, Dec. 2019
\bibitem{fl1} M. Chernysheva et al., “Carbon nanotubes for ultrafast fibre lasers,” Nanophotonics, vol. 6, no. 1, pp. 1–30, Jan. 2017, doi: 10.1515/nanoph-2015-0156.
\bibitem{fl2} C. S. Goh et al., “Femtosecond mode-locking of a ytterbium-doped fiber laser using a carbon-nanotube-based mode-locker with ultra-wide absorption band,” in (CLEO). Conference on Lasers and Electro-Optics, 2005., Baltimore, MD, USA, 2005, pp. 1644-1646 Vol. 3. doi: 10.1109/CLEO.2005.202227.
\bibitem{fl3} K. Kieu and F. W. Wise, “All-fiber normal-dispersion femtosecond laser,” Opt. Express, vol. 16, no. 15, p. 11453, Jul. 2008, doi: 10.1364/OE.16.011453.
\bibitem{fl4} Y. Z. Pan, J. G. Miao, W. J. Liu, X. J. Huang, and Y. B. Wang, “Mode-locked ytterbium fiber lasers using a large modulation depth carbon nanotube saturable absorber without an additional spectral filter,” Laser Phys. Lett., vol. 11, no. 9, p. 095105, Sep. 2014, doi: 10.1088/1612-2011/11/9/095105.

%CNT Measured
\bibitem{HC2000}NKT Photonics, “2 $\mu$m Range Hollow Core Photonic Bandgap Fiber,”HC-2000-01.
\bibitem{HC1060}NKT Photonics, “Hollow Core Photonic Bandgap Fiber for 1060nm Range Applications,”HC-1060-02.
\bibitem{hendler}A. Hendler-Neumark and G. Bisker, “Fluorescent Single-Walled Carbon Nanotubes for Protein Detection,” Sensors, vol. 19, no. 24, p. 5403, Dec. 2019, doi: 10.3390/s19245403.
\bibitem{wei} X. Wei et al., “Photoluminescence Quantum Yield of Single-Wall Carbon Nanotubes Corrected for the Photon Reabsorption Effect,” p. 23.
\bibitem{schoppler} F. Schöppler et al., “Molar Extinction Coefficient of Single-Wall Carbon Nanotubes,” J. Phys. Chem. C, vol. 115, no. 30, pp. 14682–14686, Aug. 2011, doi: 10.1021/jp205289h.
\bibitem{blanch} A. J. Blanch, C. E. Lenehan, and J. S. Quinton, “Parametric analysis of sonication and centrifugation variables for dispersion of single walled carbon nanotubes in aqueous solutions of sodium dodecylbenzene sulfonate,” Carbon, vol. 49, no. 15, pp. 5213–5228, Dec. 2011, doi: 10.1016/j.carbon.2011.07.039.
\bibitem{anson} A. Ansón-Casaos, J. M. González-Domínguez, I. Lafragüeta, J. A. Carrodeguas, and M. T. Martínez, “Optical absorption response of chemically modified single-walled carbon nanotubes upon ultracentrifugation in various dispersants,” Carbon, vol. 66, pp. 105–118, Jan. 2014, doi: 10.1016/j.carbon.2013.08.048.
\bibitem{jeong}S. H. Jeong, K. K. Kim, S. J. Jeong, K. H. An, S. H. Lee, and Y. H. Lee, “Optical absorption spectroscopy for determining carbon nanotube concentration in solution,” Synthetic Metals, vol. 157, no. 13–15, pp. 570–574, Jul. 2007, doi: 10.1016/j.synthmet.2007.06.012.
\bibitem{tsyboulski} D. A. Tsyboulski, J.-D. R. Rocha, S. M. Bachilo, L. Cognet, and R. B. Weisman, “Structure-Dependent Fluorescence Efficiencies of Individual Single-Walled Carbon Nanotubes,” Nano Lett., vol. 7, no. 10, pp. 3080–3085, Oct. 2007, doi: 10.1021/nl071561s.
%Future Work
\bibitem{ding}K. Ding et al., “Research on Narrow Linewidth External Cavity Semiconductor Lasers,” Crystals, vol. 12, no. 7, p. 956, Jul. 2022, doi: 10.3390/cryst12070956.
\bibitem{duarte} F. J. Duarte, Ed., Organic Lasers and Organic Photonics, Bristol, IOP, 2018. doi: 10.1007/978-3-662-11579-4.
\bibitem{shafer}F. P. Schäfer, Ed., Dye Lasers, vol. 1. Berlin, Heidelberg: Springer Berlin Heidelberg, 1973. doi: 10.1007/978-3-662-11579-4.
\bibitem{pierce}B. Pierce and R. Birge, “Lasing properties of several near-IR dyes for a nitrogen laser-pumped dye laser with an optical amplifier,” IEEE J. Quantum Electron., vol. 18, no. 7, pp. 1164–1170, Jul. 1982, doi: 10.1109/JQE.1982.1071672.
\bibitem{norwood} Robert A. Norwood, "Organic Photonics: Ready for Prime Time," Optics \& Photonics News 24(11), 40-47, 2013.
\bibitem{stuke}M. Stuke, Dye Lasers: 25 Years. Berlin, Heidelberg: Springer-Verlag Springer, ch.11, 2005. 

\end{thebibliography}
