%======================================================================
\chapter{Introduction}
%======================================================================
In the beginning, there was $\pi$:

\begin{equation}
   e^{\pi i} + 1 = 0  \label{eqn_pi}
\end{equation}
A \gls{computer} could compute $\pi$ all day long. In fact, subsets of digits of $\pi$'s decimal approximation would make a good source for psuedo-random vectors, \gls{rvec} . 

%----------------------------------------------------------------------
\section{Motivation}
%----------------------------------------------------------------------

See equation \ref{eqn_pi} on page \pageref{eqn_pi}.\footnote{A famous equation.}

\section{Thesis Outline}

The credo of the \gls{aaaaz} was, for several years, several paragraphs of gibberish, until the \gls{dingledorf} responsible for the \gls{aaaaz} Web site realized his mistake:

