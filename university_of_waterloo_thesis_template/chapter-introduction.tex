%======================================================================
\chapter{Introduction}
%======================================================================
With the latest developments in 1D and 2D materials, specialty optical waveguides, and fibers it has become possible to design robust small-scale nonlinear devices and sensors\cite{cusano,yamashita.tutorial} that are not burdened with taking up large amounts of space or constant realignment like traditional bulk-optical systems.
Various waveguide platforms have been developed by exploring light-guidance and confinement.
Liquid waveguides offer flexibility in tuning the intensity and spectrum filtering by filling them with liquids of various refractive-index and suspending dielectric or fluorescent particles in the liquid medium\cite{conroy, bliss, vezenov}.
This opens the possibility to develop on-chip and fiber-integrated sensors and fluorescent light sources.\\
Hollow-core photonic bandgap fibers (HCPBF) are able to confine light to an air core and offer a low-loss, high threshold powers, and tight confinement that isn't feasible in conventional optical fibers.
Because of the hollow core, it is possible to fill such fibers with gas or laser-cooled atoms and produce strong light-matter interactions\cite{bajcsy, hilton} and provide a tool for building single-photon interaction and nonlinear systems.
Some work also suggests that the optical bandgap and strong interactions are preserved in liquid-filled HCPBF \cite{antonopoulos} despite lower refractive-index contrasts.
Combining the ideas of particle suspension in liquid waveguide sensors with the desirable mode-confinement of HCPBFs, this thesis explores the interaction of light with suspended fluorescent particles in liquid-filled HCPBFs.\\ 

The structure of this thesis is as follows. In Chapter 2 the theoretical background around fiber-optic waveguides using Total-Internal-Reflection (TIR) and Hollow-Core Photonic Crystal Fibers (HCPBF) is introduced. This is followed by the derivation of the bandgap-shift equation from the refractive-index scaling laws for HCPBF in low-index contrast regions, which provides the basis for predicting the bandgap of liquid-filled hollow-core fibers.
Chapter 3 details the filling procedure for core-filled HCPBF, which transforms the HCPBF to guide light via TIR like a conventional optical fiber, and completely liquid-filled HCPBF, which should follow the scaling laws. Experimental confirmation of the predicted band-gap shift and transmission losses are done for H${}_2$O and D${}_2$O as filling materials. Chapter 4 introduces the optical properties of Indocyanine Green (ICG), a fluorescent dye used to study light-matter interactions between the particles suspended in the liquid-core and the mode of the fiber. Following the results of ICG, the optical properties of carbon nanotubes(CNTs) are introduced in Chapter 5, developing an argument for CNTs as a promising fluorescing nano-particle to be used in liquid-fiber applications. 