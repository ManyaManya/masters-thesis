\subsection{Mode Distribution}
a) For a two-level atom, the coupling constant -- g -- [Eq. 2.6] scales inversely to this 'effective mode area'  [Eq. before 2.1]  (in the interaction energy term between atom and the field).  mode is like a Gaussian [given by the mode function f(x,y) ] the photon interacts more strongly if the atom is placed in the center of the mode.)\\
effective mode area:
\begin{equation}
	A = \frac{\int dxdy|f_k(x, y)|^2}{|f_k(x_a, y_a)|^2}
\end{equation}
where $f_k(x,y)$ is the transverse mode function and $(x_a,y_a )$ is the position of the atom, is approximately constant in the range of the relevant longitudinal wave numbers\\
coupling constant:
\begin{equation}
	g_\omega = \sqrt{\frac{\omega}{4\pi\epsilon\hbar c A}}d_{eg}
\end{equation}
(2.5) dipole interaction Hamiltonian 

b) Apparently, this coupling constant term tells us parameters such as:\\
i) how likely it is for an excitation in the emitter is released into the waveguide mode $\gamma_{1D}$, versus free-space $\gamma_0$. \cite{mazoni}
\begin{equation}
	\gamma_{1D} = 2\pi g^2_{\omega_A}  = \frac{\sigma_A}{2A}\gamma_0
\end{equation}
"where the second expression directly exhibits the scaling with the transverse extension of the waveguide. It is related to the atomic radiative cross section $\sigma_A = \frac{3\lambda^2}{2\pi}$. A natural lower bound on the transverse mode size is at about $A \sim (\frac{\lambda}{2})^2$. (lowest-order mode in hollow metallic wave- guide), implying a maximum achievable coupling ratio $\gamma_{1D}/\gamma_0 \sim 1$. In the range $\sigma_A \sim A$, one has a strong waveguide-atom coupling, which is manifested by the fact that the atom dissipates its energy equally into the waveguide and the free-space ‘‘lossy’’ modes."\cite{domokos}

ii) optical depth (OD) for a single atom is (about) the ratio of $ (\gamma_{1D})/ (\gamma_{0})$ or the ratio of the cross-section area of the atom to that of the effective mode-area. 
\begin{equation}
	OD = \frac{\sigma_A}{\sigma_M} = \frac{\gamma_{1D}}{\gamma_0}
\end{equation}

c) OPTICAL DEPTH CALCULATIONS: Optical depth (OD) tells about how opaque the system is. Transmitted intensity goes by $T = exp(-OD)$. Normally, N emitters might scale linearly to give an optical depth ~$N*OD$. But now (due to its position) each emitter might have its own mode area.\\

Examples of taking this into account (with maybe slightly different conventions) for atoms are mentioned in these two references\cite{bajcsy, hilton}: 
\begin{equation}
	OD_{fiber} = \int^L_0 \int^{r}_0 n(\rho, z) OD 2\pi \rho d\rho dL
\end{equation}
$r$ and $L$ represent the radius and length of the ensemble, which in the case of solution-filled HCPCF is the radius and length of the fiber. This assumes that the particulates outside of the core do not have a significant contribution.  If the distribution of molecules is take to be uniform along the fiber length and radius of the core, then the number density will be: 
\begin{equation}
	n(\rho, z) = \begin{cases}
		0, &|z| > L/2\\
		(1/L) (1/\rho), &|z| < L/2
	\end{cases}
\end{equation}

\begin{thebibliography}{mode}
	\bibitem{domokos} P. Domokos, P. Horak, and H. Ritsch, “Quantum description of light-pulse scattering on a single atom in waveguides,” Phys. Rev. A, vol. 65, no. 3, p. 033832, Mar. 2002, doi: 10.1103/PhysRevA.65.033832.
	
	\bibitem{solano} P. Solano et al., “Optical Nanofibers: a new platform for quantum optics,” vol. 66, 2017, pp. 439–505. doi: 10.1016/bs.aamop.2017.02.003.\\
	
	\bibitem{mazoni} M. T. Manzoni, “New Systems for Quantum Nonlinear Optics,”. 2017, Thesis, p. 39-40.
	
	\bibitem{bajcsy} M. Bajcsy et al., “Laser-cooled atoms inside a hollow-core photonic-crystal fiber,” Phys. Rev. A, vol. 83, no. 6, p. 063830, Jun. 2011, doi: 10.1103/PhysRevA.83.063830. \\
	\bibitem{hilton} A. P. Hilton, C. Perrella, F. Benabid, B. M. Sparkes, A. N. Luiten, and P. S. Light, “High-efficiency cold-atom transport into a waveguide trap,” Phys. Rev. Applied, vol. 10, no. 4, p. 044034, Oct. 2018, doi: 10.1103/PhysRevApplied.10.044034. A
\end{thebibliography}