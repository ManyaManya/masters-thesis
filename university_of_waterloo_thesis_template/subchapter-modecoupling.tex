\subsection{Particle-Mode Interaction and Optical Depth}
Optical depth (OD) is a measurement of the opacity of a system, related to the transmitted intensity by $T = exp(-OD)$. With particles distributed throughout the fiber, the interaction between the beam and particles inside the fiber needs to be taken into account. When considering a single particle interacting with the mode function of a waveguide the strength of the particle interaction will depend on its position within the mode\cite{domokos, mazoni}. The effective mode area of the waveguide is then relevant only in relation to the position of the particle. 
\begin{equation}
	\sigma_M = \frac{\int dxdy|f_k(x, y)|^2}{|f_k(x_p, y_p)|^2}
\end{equation}
where $f_k(x,y)$ is the transverse mode function and $f(x_p,y_p )$ is the position of the particle. In the case of a Gaussian mode function (as would be in a HCPCF), the photon interaction with the particle will be stronger in the center of the mode and weak at the edges. The optical depth (OD) for a single particle the ratio of the scattering cross-section to that of the effective mode-area $OD =\frac{\sigma_0}{\sigma_M}$, so to find the optical depth over the entire ensemble the product of the number density of the sample and optical depth of each emitter is integrated over the volume :
\begin{equation}
	\begin{aligned}
		OD_{fiber} &= \int^{L'}_0 \int^{r'}_0 n(r, z)OD(2\pi r) dr dz 
	\end{aligned}
\end{equation}
where $r'$ and $L'$ represent the radius and length of the ensemble.  When the fibers are fully liquid cladding and core, due to the low interaction and guidance of photons in the PC structure, an approximation is made constricting the mode function strictly to the core. This simplifies the dimensions of the integration to just be the radius and length of the fiber. This assumes that the particulates outside of the core do not have a significant contribution.  If the distribution of molecules is taken to be uniform along the fiber length and radius of the core, then the number density is: 
\begin{equation}
	n(r, z) = \frac{N_{particle}}{V_{fiber}} = \frac{N_{particle}}{\pi r_{core}^2L_{fiber}} 
\end{equation}
The integral will simplify to
\begin{equation}
	\begin{aligned}
		OD_{fiber} &= \int^{L_{fiber}}_0 \int^{r_{core}}_0  n(r_{core}, L_{fiber})\sigma_0 \frac{2}{\pi w_0^2}e^{-\frac{2r^2}{w_0^2}}(2\pi r) dr dz\\
		&= N_{particle}\frac{\sigma_0}{\pi r^2_{core}}\big(1-e^{\frac{-2r_{core}^2}{w_0^2}}\big)
	\label{OD}
	\end{aligned}
\end{equation}

\subsubsection{Optical Density Calculations for ICG}
For ICG dispersed in water, molecule aggregate radii have been measured between $2nm - 200nm$ \cite{dedora}, with J-aggregates forming at radii $>50nm$\cite{weigand}. Due to the low concentration samples of dye used in out experiments, the lower range of molecule diameter is expected, meeting the Raleigh scattering approximation condition $\frac{2\pi r}{\lambda} <<1$, the scattering cross-section is
\begin{equation}
	\sigma_0 = \frac{2\pi^5 (2r_{particle})^6}{3\lambda^4}(\frac{N^2 -1}{N^2+2})^2
	\label{raleigh}
\end{equation}
where $N=\frac{n_{particle}}{n_{solvent}}$. After applying the parameters above and (\ref{raleigh}) to (\ref{OD}), the estimated concentration of ICG molecules for optically dense medium ($OD_{fiber}=1$) has a range of $N_{particle} = 1.5\times 10^8 \sim 2.0\times 10^{14}$ molecules and $N_{particle} = 8.2\times 10^7 \sim 1.1\times 10^{14}$ molecules for 1550nm and 800nm HCPCF respectively varying the ICG aggregate radii within the approximation condition. \\
\begin{figure}[!htb]
	\centering
	\foreach \x \y \z in {ICG_OD/7/5.2, OD_rate/7/5}
	{
		\begin{subfigure}[b]{0.47\textwidth}
			\includegraphics[width=\y cm,height=\z cm]{./Figures/ICG/\x.png}
			\caption{}
		\end{subfigure}
	}
	\caption{ (a)The number of molecules to create an optically dense medium as a function of average particle radius. Inset plot shows the dye concentration as a function of particle radius.(b) The rate increase in OD as the length of the fiber increases. For $OD=1$:  a sample concentration of 3.7ppm in 80nm HCPCF, $L_{fiber} = 2.28cm$ with a rate of $0.44(\frac{OD}{cm})$. For 1550nm HCPCF with a sample concentration of 4ppm. $L_{fiber} = 0.78cm$ with a rate of $1.27(\frac{OD}{cm})$}
	\label{fig:icg_od}
\end{figure}
For calculations of optical density with ICG molecules in the  1550nm HCPCF, the core radius is taken as $r_{core} = 5\pm 0.05\mu m$ with beam waist $w_0 = 4.5 \pm 0.05\mu m$. For this $L_{fiber}=1cm$  piece of fiber, the number of molecules contained in a perfectly filled core is expected to be
\begin{equation}
	N_{particle} = M_{ICG}*C*V_{fiber}=\frac{1 mol}{774.98g}*\frac{4mg}{1dm^3}*\pi(5\mu m)^2(1cm) = 2.441\times10^9 molecules
\end{equation}
Using the measured optical density in the fiber, Fig.\ref{fig:icg_fluor}b the estimated radius of the aggregate ICG molecules is $r_{particle} = 13.7 \pm 0.2$nm.
Carrying out the same calculations for the $L_{fiber}=2cm$ 800nm HCPCF, the number of molecules contained in a perfectly filled core will be $N_{particle}=2.54\times10^9$.  The fiber has a core radius $r_{core} = 3.75 \pm0.05\mu m$ with beam waist $w_0 = 2.75 \pm 0.05$ and using the measured optical density in the fiber, Fig.\ref{fig:icg_fluor_800hc}a, the estimated radius of the aggregate ICG molecules is $r_{particle} = 11.5 \pm 0.26$nm. These molecule radii are well in agreement, but since the 800nm HCPCF ICG solution is a slightly lower concentration, the average $r_{particle}$ is expected to form slightly smaller aggregates.

\clearpage