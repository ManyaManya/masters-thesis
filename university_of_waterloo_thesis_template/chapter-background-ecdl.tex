%======================================================================
\chapter{Conclusion}
%======================================================================
\subsection{Experimental Outcomes}
Determined the bandgap of liquid-filled fibers of H${}_2$O and D${}_2$O 
Measured the interaction of suspended ICG particles with the mode of the fiber
results show the potential of creating a fluorescent light source with liquid core fibers
\subsubsection{CNTs}
CNT samples from QuIN lab
measured absorption of selectively sorted sample
no fluoresence - need to do concentration-based experiemnts
furthure with nonsorted CNT sample in D${}_2$O
\begin{figure}[h]
	\centering
	\foreach \x in {OD, time}
	{ 
		\begin{subfigure}[b]{0.45\textwidth}
			\includegraphics[width=\textwidth]{./Figures/CNT_Measured/\x.png}
			\caption{}
		\end{subfigure}
		\hfil
	}
	\caption{(a) (b) }
	\label{fig:cnt_abs}
\end{figure}


\section{ECDL}
what is ECDL
fiber-integration 
diagram
advantages