\subsubsection{Future Work}
While the results of suspended ICG particles are promising and show a proof-of-concept, the progression to suspended CNTs in HCPCF is incomplete. The initial tests with a sorted CNTs sample were inconclusive and further investigation is needed to produce a fluorescent sample. Future experimental efforts will hopefully lead to uses of CNTs in fiber-integrated devices. Besides the CNT samples, one of the main challenges that remains is obtaining a consistent and high coupling to the 1550nm HCPCF. Butt-coupled mechanical splicing chips fabricated for 800nm HCPCF consistently achieve coupling around 75-80\% and robustly hold the fibers together\cite{maruf}, but the same recipe does not produce high-fidelity butt-coupled mechanical splicing chips for 1550nm HCPCF. This is another avenue that needs to be explored but we are confident that it can be solved with modest efforts.\\
After the completion of the experiments with CNT samples and improvements are made to the fiber-coupling into 1550nm HCPCF, future prospects include integrating the suspended particle-filled HCPCF into a narrow linewidth external cavity semiconductor laser (NLECSL) as a gain medium.
\begin{figure}[!htb]
	\centering
	\foreach \x  in {litrow, fbg}
	{
		\begin{subfigure}[b]{0.47\textwidth}
			\includegraphics[width=\textwidth]{./Figures/FutureWork/\x.png}
			\caption{}
		\end{subfigure}
	}
	\caption{Example NLECSL configurations integrating the liquid-HCPCF as the gain medium (a) Littrow configuration (b) Single-wavelength fiber-integrated approach with FBG.}
	\label{fig:laser}
\end{figure}
Laser-pumped dye lasers using external cavity configurations, such as the Littrow configuration depicted in Fig.\ref{fig:laser}(a) have been made using dyes in liquid forms as gain medium without optical confinement\cite{ding, duarte}. With organic dyes like ICG having wide fluorescence spectrum, a wide-range tunable laser is possible. However, open-cavity single-wavelength laser configurations 