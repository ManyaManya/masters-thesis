\subsection{Experiment Set-Up}

Preparation
1. Stock
1) Using the micropipette, measure 1ml of D2O or H2O into a vial 
2) Using the scale, measure 1mg of ICG powder
3) Pour the measured ICG into the 1ml of D2O/H2O
4) Close the vial and shake for 15 seconds to dissolve

2. Dilution
1) Measure 5ml of D2O or H2O into a vial 
2) Using the pipette, measure 10ul of the stock solution
3) Output the 10ul of stock solutions into the 100ml of D2O/H2O
4) Close the vial and shake for 15 seconds to dissolve
Used within 12 hrs of creation

\subsection{ICG in HCPCF}
Low-concentration samples of dye were prepared and filled a 800nm HCPCF core and 1550nm HCPCF core and cladding. The absorption spectrum of the dye was detected, though muddled by additional losses from the fiber, as shown in Fig. \ref{fig:icg_absp}. Additionally, there is an observed 18nm shift in the peak absorption from 778nm to 796nm in the 1550nm bandgap-shifted liquid-filled fiber, while the 800mn liquid-core fiber has an insignificant 4nm shift in peak from 775nm to 771nm.\\ 
\begin{figure}[!htb]
	\centering
	\foreach \x in {absp_800hc, absp_1550hc}
	{ 
		\begin{subfigure}[b]{0.49\textwidth}
			\includegraphics[width=\textwidth]{./Figures/ICG/\x.png}
			\caption{}
		\end{subfigure}
		\hfil
	}
	\caption{ Absorption spectrum of ICG samples (a) 2.5 ppm concentration in core of 800nm HCPCF and (b)  4 ppm concentration in core and cladding of 1550nm HCPCF. }
	\label{fig:icg_absp}
\end{figure}
For 800nm core-filled HCPCF, ICG solutions were prepared with $H_2O$ and $D_2O$ solvents, but fluorescence was only guided in the $D_2O$. In the 800HC fiber there is already significant loss coming from the narrow bandgap in combination with lower refractive index contrast of using a liquid medium, as half of the absorption spectrum is outside the bandgap; it is suspected that the absorption of effects of $H2O$ in the NIR (discussed in chapter 2) and re-absorption from the overlapping excitation-emission spectra was greater than the number of emitted photons. The fluorescence guided in the 800nm HCPCF is also influenced by the bandgap, shown in \ref{fig:icg_fluor_800hc}b,  the emission has a large shifts in peak for excitation between $745 - 775$nm - wavelengths at the edge of the bandgap and with high absorption effects- varying peak fluorescence between $800 - 820$nm while for excitation above 775nm the fluorescence stays centered at 805nm. \\
 Fluorescence was also detected in ICG-filled 1550nm HCPCF with $D2O$ as the solvent and had the best ratio of fluorescence intensity to emission intensity ('"fraction of fluorescence"). For a 4ppm ICG sample the fraction of fluorescence are compared in Fig. \ref{fig:icg_fluor} for excitation wavelengths below the emission wavelength range. The peak fluorescence in the cuvette was at 820nm but is shifted down 10nm to 810nm in the fiber and the fraction of fluorescence in 1550nm HCPCF $\sim35$x greater than that measured through the cuvette. For the 3.7ppm sample in 800nm HCPCF measured similar fraction of fluorescence to the cuvette sample at the excitation wavelength of peak absorption(778nm), but the exact source of the large difference in fraction of fluorescence in the 1550nm and 800nm HCPCF/cuvette are not  clear. 
\begin{figure}[!htb]
	\centering 
	\foreach \x \y \z in {diff_max_pl/8/6, OD/7/5.5, fluor_1550hc/7/5.5, fluor_cuvette/7.5/5.5}
	{ 
		\begin{subfigure}[b]{0.47\textwidth}
			\includegraphics[width=\y cm,height=\z cm]{./Figures/ICG/\x.png}
			\caption{}
		\end{subfigure}
	}
	\caption{(a) The maximum fraction of fluorescence is plotted against excitation wavelength for a 4ppm ICG sample in a 1cm piece of 1550nm HCPCF and 1cm cuvette. The maximum fraction of fluorescence of the ICG in the cuvette is only 4\% of that measured in fiber. (b)  The optical density at each excitation wavelength. The fraction of fluorescence spectrum of the 4ppm ICG solution in (c) 1550nm HCPCF (d) a cuvette. }
	\label{fig:icg_fluor}
\end{figure}

At the maximum absorption wavelength, measurements of the the fraction of fluorescence and output power at the excitation wavelength are  . Overall, the collection efficiency is in the range of ~0.001\%, for milliwatts of pump power nanowatts of power at the emission frequency.
\begin{figure}[!htb]
	\centering
	\foreach \x \y \z in {wl_fluor_od_800hc/5.5/5, fluor_800hc/5.5/5, power_fl_absp_800hc/5.5/5.2}
	{ 
		\begin{subfigure}[b]{0.32\textwidth}
			\includegraphics[width=\y cm,height=\z cm]{./Figures/ICG/\x.png}
			\caption{}
		\end{subfigure}
	}
	\caption{ Measurements of 3.7ppm ICG sample in a 2cm piece of core-filled 800nm HCPCF (a) The maximum fraction of fluorescence and optical density against excitation wavelength. (b) The fraction of fluorescence spectrum (c) Measured output peak power and fractional fluorescence as a function of input power. }
	\label{fig:icg_fluor_800hc}
\end{figure}

\clearpage
\subsubsection{Optical Density Calculations}
For calculations of optical density within the HCPCF, the core radius will be  $r_{core} = 5\pm 0.05\mu m$ with beam waist $w_0 = 4.5 \pm 0.05\mu m$ for 1550nm HCPCF and $r_{core} = 3.75 \pm0.05\mu m$ with beam waist $w_0 = 2.75 \pm 0.05$. For ICG dispersed in water, molecule aggregate radii have been measured between $2nm - 200nm$ \cite{dedora} with J-aggregates forming at radii $>50nm$\cite{weigand}. Due to the low concentration samples of dye used in out experiments, the lower range of molecule diameter is used, meeting the Raleigh scattering approximation condition $\frac{2\pi r}{\lambda} <<1$, the scattering cross-section is 
\begin{equation}
	\sigma_0 = \frac{2\pi^5 (2r_{particle})^6}{3\lambda^4}(\frac{N^2 -1}{N^2+2})^2
	\label{raleigh}
\end{equation}
where $N=\frac{n_{particle}}{n_{solvent}}$. Applying the parameters above and  (\ref{raleigh}) to (\ref{OD}) the estimated concentration of ICG molecules for optically dense medium ($OD_{fiber}=1$) has a range of $N_{particle} = 1.5\times 10^8 \sim 2.0\times 10^{14}$ molecules and $N_{particle} = 8.2\times 10^7 \sim 1.1\times 10^{14}$ molecules for 1550nm and 800nm HCPCF respectively varying the ICG aggregate radii within the approximation condition. \\

For the $L_{fiber}=1cm$  piece of fiber, the number of molecules contained in a perfectly filled core will be
\begin{equation}
	N_{particle} = M_{ICG}*C*V_{fiber}=\frac{1 mol}{774.98g}*\frac{4mg}{1dm^3}*\pi(5\mu m)^2(1cm) = 2.441\times10^9 molecules
\end{equation} 
Using the measured optical density in the fiber, Fig. \ref{fig:icg_fluor}b the estimated radius of the aggregate ICG molecules is $r_{particle} = 13.7 \pm 0.2nm$.  

\clearpage