\subsection{Experiment Set-Up}
The experimental set-up for the dye-filled HCPCF is very similar to that of the previously introduced HCPCF set-up, but includes a few additions. A fiber beam splitter was added between the fiber patch coming from the laser couplet and the bare fiber to monitor input power and wavelength. For fluorescence detection out of the fiber, a front-face fluorescence collection scheme was used and a long pass filter was placed in the path in order to filter out the incident light from the fluorescence. 
Additionally, a fiber clamp was placed on a translation stage to control the position and angle at which the bare fiber is positioned in the mechanical splicer chip. Due to tolerance issues in the mechanical splitter chips made for 1550nm HCPCF. The prior development of these chips done by lab members of NPQO\cite{maruf} tuned the fabrication recipe for coupling between 130$\mu$m to 125$\mu$m diameter fibers, specifically measuring the coupling between 800nm HCPCF to PM780HP fibers, resulting in a production of chips highly consistent in diameter. The 1550nm HCPCF however, has a diameter of 120$\mu$m. Mechanical splicer chips of 120$\mu$m to125$\mu$m diameter fibers are fabricated along with the 130$\mu$m to 125$\mu$m, but they are not of consistent quality and the majority of chips are over-exposed leading to the fibers to be loose and not level with each other in the chip. This resulteed in average coupling rates of $>10\%$ if supported by the chip structure only.
\begin{figure}[!htb]
	\centering
	\foreach \x \y in {ICG_Spec_SetUp.png/0.39,ICG_Cuvette_SetUp.png/0.39, ICG_Color.jpg/0.1}
	{ 
		\begin{subfigure}[b]{\y\textwidth}
			\includegraphics[width=\textwidth]{./Figures/ICG/\x}
			\caption{}
		\end{subfigure}
		\hfil
	}
	\caption{ (a)Optical set-up to measuring output of the dye-filled fiber. The path to the CCD camera is used to monitor the modeshape coming out of the fiber. The path to the spectrometer (Case A) is used to measure the fluorescence spectrum and efficiency. The path to the photodiode (Case B) is used to measure the optical density of the fiber. (b)Optical set-up to measuring output of the dye in a 1cm cuvette. (c)Color and opacity difference between the stock solution (left) and diluted solution(right).}
	\label{fig:icg_setup}
\end{figure}
\subsubsection{Sample Preparation}
ICG powder was purchased from MP Biomedicals in quantities of 5mg per vial. A stock solution was made by dissolving 5mg of powder in 5mL of H${}_2$O, then a low-concentration solution is made by diluting 10$\mu$L of stock solution into 2.5mL of either D${}_2$O/H${}_2$O.  In a second approximate method for making low-concentration solutions, the tip of a syringe needle was use to scoop up a small amount of dye and then dissolve it in 5-10mL of D${}_2$O/H${}_2$O, working by eye based on the hue and opacity of the dye sample (See Fig.\ref{fig:icg_setup}(c)). The absorption of the sample was then measured in a cuvette and fitted to the concentration-based absorption cross-section data to find the concentration of the sample.
\clearpage
\subsection{ICG in HCPCF}
Low-concentration samples of dye were prepared and used it to fill a 800nm HCPCF core and 1550nm HCPCF core and cladding. The measured absorption cross-section of the dye, shown in Fig.\ref{fig:icg_absp}, presents additional absorption effects caused by the fiber, notably the narrower bandgap of the 800nm HCPCF. Additionally, there is an observed 18nm shift in the peak absorption from 778nm to 796nm in the 1550nm bandgap-shifted liquid-filled fiber, while the 800mn liquid-core fiber has an insignificant 4nm shift in peak from 775nm to 771nm.\\
\begin{figure}[!htb]
	\centering
	\foreach \x in {absp_800hc, absp_1550hc}
		{
			\begin{subfigure}[b]{0.49\textwidth}
				\includegraphics[width=\textwidth]{./Figures/ICG/\x.png}
				\caption{}
			\end{subfigure}
			\hfil
		}
	\caption{ Absorption spectrum of ICG samples (a) 2.5 ppm concentration in core of 800nm HCPCF and (b) 4 ppm concentration in core an
		d cladding of 1550nm HCPCF. }
	\label{fig:icg_absp}
\end{figure}

For 800nm core-filled HCPCF, ICG solutions were prepared with H${}_2$O and  D${}_2$O solvents, but fluorescence was only guided in the D${}_2$O. In the 800HC fiber there was already significant loss coming from the narrow bandgap in combination with lower refractive index contrast of using a liquid medium, as half of the absorption spectrum is outside the bandgap; The absorption of effects of H${}_2$O in the NIR (discussed in chapter 2) and re-absorption from the overlapping excitation-emission spectra is suspected to be greater than the number of emitted photons. The fluorescence guided in the 800nm HCPCF was also influenced by the bandgap, shown in Fig.\ref{fig:icg_fluor_800hc}b. The emission had a large shifts in peak for excitation between $745 - 775$nm - wavelengths at the edge of the bandgap and with high absorption effects - varying peak fluorescence between $800$ and $820$nm while for excitation above 775nm the fluorescence stayed centered at 805nm. \\
Fluorescence was also in ICG-filled 1550nm HCPCF with D${}_2$O as solvent, which had the best ratio of fluorescence intensity to emission intensity (``fraction of fluorescence''). For a 4ppm ICG sample the fraction of fluorescence are compared in Fig.\ref{fig:icg_fluor} for excitation wavelengths below the emission wavelength range.
The peak fluorescence in the cuvette was at 820nm, but was shifted down 10nm to 810nm in the fiber and the fraction of fluorescence in 1550nm HCPCF was $\sim35$x greater than that measured through the cuvette.
For the 3.7ppm sample in core-filled 800nm HCPCF similar fraction of fluorescence to the cuvette sample were measured at the excitation wavelength of peak absorption(778nm).

\begin{figure}[!htb]
	\centering
	\foreach \x \y \z in {diff_max_pl/8/6, OD/7/5.5, fluor_1550hc/7/5.5, fluor_cuvette/7.5/5.5}
		{
			\begin{subfigure}[b]{0.47\textwidth}
				\includegraphics[width=\y cm,height=\z cm]{./Figures/ICG/\x.png}
				\caption{}
			\end{subfigure}
		}
	\caption{(a) The maximum fraction of fluorescence is plotted against excitation wavelength for a 4ppm ICG sample in a 1cm piece of 1550nm HCPCF and 1cm cuvette. The maximum fraction of fluorescence of the ICG in the cuvette is only 4\% of that measured in fiber. (b)  The optical density at each excitation wavelength. The fraction of fluorescence spectrum of the 4ppm ICG solution in (c) 1550nm HCPCF (d) a cuvette. }
	\label{fig:icg_fluor}
\end{figure}
\clearpage
At the maximum absorption wavelength, the fraction of fluorescence and output power at the excitation wavelength are measured as a function of the input power, shown in Fig.\ref{fig:icg_fluor_800hc}c. The fraction of fluorescence peaks at an input power of $50\mu W$, while the output power increases logarithmically and appears to approaching a limit on the transmission. Overall, the fluorescence efficiency is of $0.00051\%$ in the 800nm HCPCF $\lambda_{ex}=778$ and for 1550nm HCPCF $0.014\%$ at $\lambda_{ex}=740$nm, which is comparable to other optofluidic waveguides \cite{vezenov}.
>>>>>>> Stashed changes
\begin{figure}[!htb]
	\centering
	\foreach \x \y \z in {wl_fluor_od_800hc/5.5/5, fluor_800hc/5.5/5, power_fl_absp_800hc/5.5/5.2}
		{
			\begin{subfigure}[b]{0.32\textwidth}
				\includegraphics[width=\y cm,height=\z cm]{./Figures/ICG/\x.png}
				\caption{}
			\end{subfigure}
		}
	\caption{ Measurements of 3.7ppm ICG sample in a 2cm piece of core-filled 800nm HCPCF (a) The maximum fraction of fluorescence and optical density against excitation wavelength. (b) The fraction of fluorescence spectrum (c) Measured output peak power and fractional fluorescence as a function of input power. }
	\label{fig:icg_fluor_800hc}
\end{figure}
\clearpage